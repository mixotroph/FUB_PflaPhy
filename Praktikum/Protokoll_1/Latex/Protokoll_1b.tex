\documentclass[11pt,a4paper,DIV=10,]{scrartcl}
%\usepackage[latin1]{inputenc}
\usepackage[utf8]{inputenc}
\usepackage[ngerman]{babel}
\usepackage{amsmath}
\usepackage{amsfonts}
\usepackage{amssymb}
\usepackage{fancybox}
\usepackage{multicol}
\usepackage{graphicx}
\usepackage{color}
\usepackage{colortbl}
% Define user colors using the RGB model
\definecolor{dunkelgrau}{rgb}{0.8,0.8,0.8}
\definecolor{hellgrau}{rgb}{0.95,0.95,0.95}

\usepackage[normal,font={small,color=black}, labelfont=bf,figurename=Abb.]{caption}
\usepackage{float}
\usepackage{cite}
\usepackage{url}
\bibliographystyle{unsrtnat}
\usepackage[numbers]{natbib}
\usepackage[T1]{fontenc}
\usepackage{booktabs}

\begin{document}
\onecolumn
\subsection*{Versuchsprotokoll zum 1. Kurstag }
\section*{Hemmung der Blattseneszenz durch Cytokinin anhand des Chlorophyll-Abbaus}
\textbf{Christoph van Heteren-Frese\footnotemark[1]} \\[0.1cm]
\footnotemark[1]Freie Universität Berlin\\[0.2cm]
Eingereicht am 13.11.2012\\
\hrule

\section*{Einleitung}    
Das vorliegende Protokoll wurde von den Autoren im Rahmen des pflanzenphysiologischen Grundpraktikums als Bestandteil des Moduls Physiologie und Biochemie der Pflanzen und Tiere der Freien Universität Berlin verfasst. 


\section*{Material und Methoden}
%\input{Material_und_Methoden}
%\section*{Ergebnisse und Diskussion}

\section*{Ergebnisse}

% Hier beginnt das erste Bild 
\begin{figure}[H]
\center
\captionsetup{width=1\textwidth}	
\includegraphics[width=1\textwidth]{Abbildungen/Rplot.eps}
\caption{Gewichtszu- bzw. Abnahme..}
\label{v1}
\end{figure}

\section*{Diskussion}

\end{document}