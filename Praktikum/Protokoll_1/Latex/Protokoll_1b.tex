\documentclass[11pt,a4paper,DIV=10,]{scrartcl}
%\usepackage[latin1]{inputenc}
\usepackage[utf8]{inputenc}
\usepackage{lmodern}
\usepackage[ngerman]{babel}
\usepackage{amsmath}
\usepackage{amsfonts}
\usepackage{amssymb}
\usepackage{fancybox}
\usepackage{multicol}
\usepackage{graphicx}
\usepackage{color}
\usepackage{colortbl}
% Define user colors using the RGB model
\definecolor{dunkelgrau}{rgb}{0.8,0.8,0.8}
\definecolor{hellgrau}{rgb}{0.95,0.95,0.95}
\usepackage{booktabs}
\usepackage[normal,font={small,color=black}, labelfont=bf,figurename=Abb.]{caption}
\usepackage{float}
\usepackage{cite}
\usepackage{url}
\bibliographystyle{unsrtnat}
\usepackage[numbers]{natbib}
\usepackage[T1]{fontenc}
%
\begin{document}
\onecolumn
\subsection*{Versuchsprotokoll zum 1. Kurstag }
\section*{Zum Wasserhaushalt von Pflanzen -- Bestimmung des osmotischen Potentials bei \textit{Solanum Tuberosum}}
\textbf{Christoph van Heteren-Frese\footnotemark[1], Jana Jerosch\footnotemark[1], Nora Pauli\footnotemark[1]} \\[0.1cm]
\footnotemark[1]Freie Universität Berlin\\[0.2cm]
Eingereicht am 13.11.2012\\
\hrule
%
\section*{Einleitung}    
Das vorliegende Protokoll wurde von den Autoren im Rahmen des pflanzenphysiologischen Grundpraktikums als Bestandteil des Moduls Physiologie und Biochemie der Pflanzen und Tiere der Freien Universität Berlin verfasst. 


\section*{Material und Methoden}
\subsection*{Material}
Der Rest der bereits im vorigen Versuch benutzten Kartoffel
\subsection*{Kryoskopie}
Da der Gefrierpunkt eines Lösungsmittels durch Zugabe von löslichen Substanzen erniedrigt wird, 
\citep[vgl.][S. 261]{strasburger_lehrbuch_2012}
%\input{Material_und_Methoden}
%\section*{Ergebnisse und Diskussion}

\section*{Ergebnisse}
Die Messung ergab einen Wert von 346 mOsm/kg. Durch die Verhältnisgleichung
\begin{equation}
\dfrac{-24,8 \cdot 10^5 \textrm{~Pa}}{1 \textrm{~Osm} } = \dfrac{x}{0,346 \textrm{~Osm}}
\end{equation}
lässt sich das osmotische Potential \textbf{des Zellinhalts} berechnen:
\begin{equation}
x = \dfrac{-24,8 \cdot 10^5 \textrm{~Pa} \cdot 0,346 \textrm{~Osm}}{1 \textrm{~Osm} } = -0,858 \textrm{~MPa}
\end{equation}
Mit dem Ergebnis der Kompensationsmethode lässt sich das Druckpotential (\textbf{Turgor}) berechnen:
\begin{equation}
\Psi_p=\Psi_W-\Psi_S=-0,56 \textrm{~MPa} - (-0,86 \textrm{~MPa})= 0,3 \textrm{~MPa}
\end{equation}
\section*{Diskussion}
Das Ergebniss der Messung entspricht den Erwartungen.

Der Vergleich der Werte..
\bibliographystyle{agsm}
\bibliography{pflaphy}
\end{document}