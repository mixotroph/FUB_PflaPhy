\documentclass[11pt,a4paper,DIV=10,]{scrartcl}
%\usepackage[latin1]{inputenc}
\usepackage[utf8]{inputenc}
\usepackage[ngerman]{babel}
\usepackage{amsmath}
\usepackage{amsfonts}
\usepackage{amssymb}
\usepackage{fancybox}
\usepackage{multicol}
\usepackage{graphicx}
\usepackage{color}
\usepackage{colortbl}
% Define user colors using the RGB model
\definecolor{dunkelgrau}{rgb}{0.8,0.8,0.8}
\definecolor{hellgrau}{rgb}{0.95,0.95,0.95}

\usepackage[normal,font={small,color=black}, labelfont=bf,figurename=Abb.]{caption}
\usepackage{float}
\usepackage{cite}
\usepackage{url}
\bibliographystyle{unsrtnat}
\usepackage[numbers]{natbib}
\usepackage[T1]{fontenc}

\begin{document}
\onecolumn
\subsection*{Versuchsprotokoll zum 1. Kurstag }
\section*{Auslösung und Erregungsleitung von Aktionpotentialen im Bauchmark des Regenwurms, \textit{Lumbricus Terrestris}}
\textbf{ Manon Couvignou\footnotemark[1], Christoph van Heteren-Frese\footnotemark[1], Jennifer Speier\footnotemark[1]} \\[0.1cm]
\footnotemark[1]Freie Universität Berlin\\[0.2cm]
Eingereicht am 13.11.2012\\
\hrule

\section*{Einleitung}    
Das vorliegende Protokoll wurde von den Autoren im Rahmen des neurologischen Grundpraktikums als Bestandteil des Moduls Verhaltens- und Neurobiologie der Freien Universität Berlin verfasst. Es beschränkt sich bewusst auf den Material und Methodenteil sowie auf die Errgebnissauswertung  mit den zugehörigen Erläuterungen.

% Hier beginnt das erste Bild 
\begin{figure}[H]
\center
\captionsetup{width=1\textwidth}	
\includegraphics[width=1\textwidth]{Abbildungen/Rplot.eps}
\caption{Membranpotenzial und extrazelluläre Kaliumkonzentration in halblogarithmischer Darstellung. Die Werte beider Graphen wurden mit der Simulationssoftware NEUROSIM ermittelt. Bei geringer Kaliumkonzentration weichen die Meßwerte von einander ab, da die Membran auch für Natrium permeabell ist.}
\label{v1}
\end{figure}

\section*{Material und Methoden}
%\input{Material_und_Methoden}
%\section*{Ergebnisse und Diskussion}

\end{document}